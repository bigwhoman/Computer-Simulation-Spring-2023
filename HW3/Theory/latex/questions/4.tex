\smalltitle{سوال 4}
\begin{enumerate}
    \item \phantom{text}
          \\
          در این روش یک دنباله تصادفی از اعداد صحیح بین 0 , m-1 تولید می‌کنیم.
          \\
          سپس مطابق فرمول زیر در هر مرحله عدد تصادقی جدید را بدست آورده و نهایتا برای اینکه اعداد تصادفی ما بین ۰ و ۱ هستند، عدد بدست آمده را بر m تقسیم می‌کنیم.
          \\
          در نوشتار زیر، $X_i$ همان اعداد تصادفی تولید شده هستند، Seed اولین عددی است که به فرمول می‌دهیم، a مضرب اعداد ما است، c میزان اضافه شده در هر قدم است و m نیز پیمانه ما است.
          \begin{latin}
              $X_{i+1} = (aX_i+c) \text{ mod } m$
              \\
              $X_0 = Seed$
              \\
              $Random_i=\frac{X_i}{m}$
          \end{latin}
    \item \phantom{text}
          \\
          وقتی در LCM می‌خواهیم اعداد را تولید کنیم، نیاز به یک عدد اولیه به نام Seed داریم. حال دنباله تولید شده
          نهایتا برابر $X_0,\cdots,X_n$ خواهد بود که پس از $X_n$ دوباره دنباله از اول تولید خواهد شد. مفهوم random number stream
          این است که یکی از اعداد این دنباله را به عنوان Seed در نظر بگیریم و سپس اعداد را تولید کنیم تا نهایتا به عدد دیگری در انتها برسیم.
          هر استریم را می‌توان به عنوان خروجی تولیدکننده‌های مجزا در نظر گرفت.
    \item \phantom{text}
          \\
          برای بدست آوردن ماکسیمم پریود باید اولا که c , m نسبت به یکدیگر اول باشند.
          \\
          ثانیا a-1 باید بر تمامی عوامل اول m بخش‌پذیر باشد و اگر m بر ۴ بخش‌پذیر بود a-1 نیز باید بر ۴ بخش‌پذیر باشد.
          \\
          حال برای اینکه مقادیر درست را برای a,c,m انتخاب کنیم باید چند شرط را رعایت کنیم.
          \\
          m باید یک عدد اول خیلی بزرگ و یا توانی از یک عدد اول باشد.

    \item \phantom{text}
          \\
          مزایا :‌
          \\
          \begin{enumerate}
              \item سرعت محاسبه اعداد با این روش بسیار سریع است.
              \item از نظر حافظه نیاز به حافظه زیادی ندارد
              \item پارامترهای آن را به راحتی می‌توان customize کرد
              \item برای پیاده‌سازی و استفاده روش راحت و سرراستی است.
          \end{enumerate}
          معایب :‌
          \\
          \begin{enumerate}
            \item در این روش اعداد نیمه تصادفی تولید می‌شوند و نه لزوما کاملا تصادفی که این باعث می‌شوند با یکدیگر یک همبستگی داشته باشند.
            \item پریود اعداد تولید شده در این روش محدود است و پس از مدتی اعداد تکراری تولید می‌شوند.
            \item این روش به شدت به پارامترها حساس است و در صورتی که پارامترهای اشتباه بدهیم اعداد تولید شده اعداد خوبی نخواهند بود.
            \item از نظر امنیتی امن نیست چون الگوی قابل شناسایی دارد.
          \end{enumerate}
\end{enumerate}
