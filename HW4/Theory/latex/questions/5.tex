\smalltitle{سوال 5}
\begin{enumerate}
    \item \phantom{text}

می‌دانیم در توزیع پوآسون برای تخمین پارامتر می‌توانیم از میانگین وزن‌دار استفاده کنیم.
\\
\begin{latin}
    \noindent
    $\alpha = \bar{X} = \frac{\sum_{j=1}^{c}f_jm_j^2}{n}$ \\\\
    $\alpha = \frac{30 * 0 + 45 * 1 + 15 *2 + 7 * 3 + 2 * 4 + 5 * 1}{100} = 1.09$ \\\\
    $E_i = np(x) = n\frac{e^{-\alpha}\alpha^x}{x!}$\\\\
\end{latin}
\begin{latin}
    \noindent
    \begin{tabular}{ |p{3cm}||p{3cm}|p{3cm}|p{3cm}|  }
        \hline
        \multicolumn{4}{|c|}{Test table}                    \\
        \hline
        $x_i$ & $O_i$ & $E_i$ & $\frac{(O_i-E_i)^2}{E_i}$   \\
        \hline
        0     & 30    & 33.62   & 0.389                        \\
        1     & 45    & 36.64   & 1.9                         \\
        2     & 15    & 19.97   & 1.236                         \\
        3     & 7    & 7.25   &                          \\
        4     & 2    & 1.97   & 0.012                         \\
        5     & 1    & 0.43   &                          \\
        \hline
              & 100      &  99.88      & 3.537                          \\
        \hline
    \end{tabular}
    \\\\
    Degree of freedom = $k-s-1=4 - 1 -1 = 2$ \\\\
    $\chi_0^2 = 3.537 < \chi_{0.05,2}^2=5.99 \rightarrow$ Hypothesis not rejected.
\end{latin}

\item \phantom{text}
\\
در این قسمت نیز با توجه به پارامتر داده‌شده جدول مورد نظر را کامل می‌کنیم.
\\
\begin{latin}
    \noindent
    \begin{tabular}{ |p{3cm}||p{3cm}|p{3cm}|p{3cm}|  }
        \hline
        \multicolumn{4}{|c|}{Test table}                    \\
        \hline
        $x_i$ & $O_i$ & $E_i$ & $\frac{(O_i-E_i)^2}{E_i}$   \\
        \hline
        0     & 30    & 22.31   & 2.648                        \\
        1     & 45    & 33.46  & 3.97                        \\
        2     & 15    & 25.10   & 4.06                         \\
        3     & 7    & 12.55   &  2.45                        \\
        4     & 2    & 4.70   & 1.582                        \\
        5     & 1    & 1.41  &                          \\
        \hline
              & 100      &  99.53      & 14.71                         \\
        \hline
    \end{tabular}
    \\\\
    Degree of freedom = $k-s-1=5 - 1 -1 = 3$ \\\\
    $\chi_0^2 = 14.71 > \chi_{0.05,3}^2=7.81 \rightarrow$ Hypothesis rejected.
\end{latin}



\end{enumerate}