\smalltitle{سوال 2}
\begin{enumerate}
  \item \phantom{text}
        \\
        برای این کار ابتدا تابع توزیع احتمال را بدست می‌آوریم.\\
        \begin{latin}
          $P[X] = \begin{cases}
              \frac{50}{110} & X = apple     \\
              \frac{30}{110} & X = banana   \\
              \frac{20}{110} & X = orange    \\
              \frac{10}{110} & X = pineapple \\
            \end{cases}$
        \end{latin}
        حال تابع توزیع تجمعی را بدست می‌آوریم.
        \\
        \begin{latin}
          $
            CDF = F(x) = \begin{cases}
                \frac{50}{110} & X = apple \\
                \frac{80}{110} & X = banana \\
                \frac{100}{110} & X = orange \\
                \frac{110}{110} & X = pineapple \\
            \end{cases}
          $
        \end{latin}
      \item \phantom{text}
      \\
      حال بررسی می‌کنیم که هر کدام از اعداد در کدام بازه قرار می‌گیرند.\\
      \begin{latin}
        $\begin{cases}
          0 < 0.3,0.45 < \frac{50}{110} = 0.454 \rightarrow apple \\
          0.454 < 0.6 < \frac{80}{110} = 0.727 \rightarrow banana \\
          0.727 < 0.8 < \frac{100}{110} = 0.909 \rightarrow orange \\
          0.909 < 0.95 < \frac{110}{110} = 1.000 \rightarrow pineapple
        \end{cases}$
      \end{latin}
\end{enumerate}
