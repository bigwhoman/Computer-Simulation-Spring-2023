\smalltitle{سوال 1}


برای این کار طبق قرمول گفته شده در اسلایدها عمل می‌کنیم.
\\
ابتدا باید cdf تابع گفته‌شده را بدست آوریم، سپس باید تابع معکوس آنها را برای یافت خواسته سوال استفاده کنیم.
\\

\begin{latin}
    \noindent
    $CDF =
    \begin{cases}
        0 & x < -2 \\\\
        \int_{-2}^{x} \frac{3t^2}{16}   \,dt = \frac{1}{16}(8 + x^3) & -2 \leq x \leq 0 \\\\
        \frac{1}{2} + \int_{0}^{x} e^{-2t}  \,dt = 1 - \frac{e^{-2 x}}{2} & 0 \leq x \\\\

    \end{cases}$\\\\


    \noindent
    $x=F^{-1}(R_i)\rightarrow$ \\\\
    $F^{-1}(R_i) = \begin{cases}
            \text{ }\sqrt[3]{16R-8} & 0 \leq R < \frac{1}{2} \\\\
            \frac{\ln(2(1-R))}{-2}                               & \frac{1}{2} \leq R \leq 1
        \end{cases} $
\end{latin}
حال مطابق چیزی که دیدیم در صورتی که اعداد تصافی R را در فرمول مناسب جای‌گذاری کنیم، مطابق توزیع گفته شده random variable خواهیم تولید کرد.

