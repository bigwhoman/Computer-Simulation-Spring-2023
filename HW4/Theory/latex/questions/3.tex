\smalltitle{سوال 3}
\\
می‌دانیم که توزیع negative bionomial عملا جمع چندین متغیر تصادفی با توزیع هندسی است.\\
مثلا فرض کنید $X \thicksim  NB(r,p)$ که یعنی X می‌گوید چه تعداد اقدام ناموفق تا قبل از دیدن r موفقیت داشته‌ایم که در هر دور احتمال موفقیت نیز p است.
\begin{latin}
  $X = Y_1 + Y_2 + \cdots + Y_r, Y \thicksim Geometric(p)$\\\\
  $CDF(Y) = F(Y=y) = 1 - (1-p)^y = R \xrightarrow[]{\text{Inverse Transform}} y_i = \log_{1-p}(1-R_i)$ \\\\
  $X = \sum_{i=1}^{r}y_i = \sum_{i=1}^{r}\log_{1-p}(1-R_i) = \sum_{i=1}^{r}\log_{1-p}(R'_i) =\log_{1-p}(\Pi_{i=1}^{r}R'_i) \leq 1 < \log_{1-p}(\Pi_{i=1}^{r+1}R'_i) \\\\
   \rightarrow 1 - p < \Pi_{i=1}^{r+1}R'_i$
\end{latin}
می‌توان از روش دیگر استفاده کرد که با استفاده از توزیع نمایی یک عدد مثل Y تولید بکنیم و در مرحله بعدی این عدد را در توزیع negative bionomial قرار دهیم حال سپس در هر مرحله توزیع NB را بر geometric تقسیم کرده و اگر عدد تولید شده رندوم ما از این مقدار کوچکتر بود آنرا قبول می‌کنیم و در غیر این صورت دوباره مراحل را از اول با اعداد جدید انجام می‌دهیم.